\sloppy

% Настройки стиля ГОСТ 7-32
% Для начала определяем, хотим мы или нет, чтобы рисунки и таблицы нумеровались в пределах раздела, или нам нужна сквозная нумерация.
\EqInChapter % формулы будут нумероваться в пределах раздела
\TableInChapter % таблицы будут нумероваться в пределах раздела
\PicInChapter % рисунки будут нумероваться в пределах раздела

% Добавляем гипертекстовое оглавление в PDF
\usepackage[
    bookmarks=true, colorlinks=true, unicode=true,
    allcolors=black,
]{hyperref}

\AfterHyperrefFix

\usepackage{microtype}% полезный пакет для микротипографии, увы под xelatex мало чего умеет, но под pdflatex хорошо улучшает читаемость

% Тире могут быть невидимы в Adobe Reader
\ifInvisibleDashes
    \MakeDashesBold
\fi

\usepackage{graphicx}

% Пакет Tikz
\usepackage{tikz}
\usetikzlibrary{arrows,positioning,shadows}

% Произвольная нумерация списков.
\usepackage{enumerate}

% ячейки в несколько строчек
\usepackage{multirow}

% itemize внутри tabular
\usepackage{paralist,array}

\setlength{\parskip}{0ex} % разрыв между абзацами

% Пакет для генерации случайного текста
\usepackage{blindtext}

% Центрирование подписей к плавающим окружениям
%\usepackage[justification=centering]{caption}

\usepackage{newfloat}
\DeclareFloatingEnvironment[
placement={!ht},
name=Equation
]{eqndescNoIndent}
\edef\fixEqndesc{\noexpand\setlength{\noexpand\parindent}{\the\parindent}\noexpand\setlength{\noexpand\parskip}{\the\parskip}}
\newenvironment{eqndesc}[1][!ht]{%
    \begin{eqndescNoIndent}[#1]%
\fixEqndesc%
}
{\end{eqndescNoIndent}}

% Подключаем пакет BibLaTeX для работы с библиографией
\usepackage[
parentracker=true,
style=gost-numeric,
bibstyle=gost-numeric,
language=auto,
autolang=other,
backend=biber,
sorting=none]{biblatex}
% Добавляем библиографию
\addbibresource{rpz.bib}
% Корректная обработка даты обращения в формате дд.мм.гг
\DeclareSourcemap{
    \maps{
        \map[overwrite]{
            \step[fieldsource=urldate,
            match=\regexp{([0-9]{2})\.([0-9]{2})\.([0-9]{4})},
            replace={$3-$2-$1},
            final]
        }
    }
}
% Убираем курсив из списка источников
\makeatletter
\renewrobustcmd*{\mkbibemph}{}
\protected\long\def\blx@imc@mkbibemph#1{#1}
\makeatother

